\documentclass{prometheus_cv}
\usepackage[utf8]{inputenc}
\usepackage[left=1.5in, right=0.5in, bottom=0.25in, top=0.25in]{geometry}		% width=5.3in, height=10in, top margin=2cm on each page

\usepackage{xcolor}																				  % define some colors
\definecolor{highlight}{HTML}{283559}
\definecolor{highlight2}{HTML}{1A2640}
\definecolor{link}{HTML}{9b3154}

\usepackage{lipsum}
\usepackage{siunitx}							% package to properly set units
\usepackage{fontawesome5}						% package for icons (see list of available icons here: http://mirrors.ibiblio.org/CTAN/fonts/fontawesome5/doc/fontawesome5.pdf)
\usepackage[super]{nth}							% when you write \nth{2} you get a nice superscript
\usepackage[								% coloring of links
	colorlinks = true,
	linkcolor = highlight,
	urlcolor = highlight,
	citecolor= highlight
]{hyperref}

\usepackage{fontspec}							% package to change fonts				
\setmainfont[]{Charter}
\newfontfamily\GaramondLight{Charter}
\newcommand\textlf[1]{{\GaramondLight#1}}

\newcommand{\highlight}[1]{\textcolor{highlight}{\textbf{#1}}}		% highlight text as bold and with the highlight color defined above
\newcommand{\ec}{\textsuperscript{\textdagger}}						% Equal contribution dagger

% define the header and (not) footer %
\usepackage{fancyhdr}																		  
\fancyhf{}
\rhead{Curriculum Vit\ae}
\lhead{Prometheus Fire}
\rfoot{Page \thepage}
\renewcommand{\headrulewidth}{0.5pt}
\renewcommand{\footrulewidth}{0pt}

\begin{document}
\thispagestyle{empty}					% Turn off header and footer for the first page
\pagestyle{fancy}			 		% For the rest of the pages switch to the fancy page style defined just above the document begin

%%%%%%%%%%%%%%%%%%%%%%%%%%%%%%
%%%%%%%%%%%% TITLE %%%%%%%%%%% 
%%%%%%%%%%%%%%%%%%%%%%%%%%%%%%
\centering 

\begin{Large} 
   Richard Posert
\end{Large}

\vspace*{0.25em}
\begin{scshape}
   \begin{footnotesize}
		 \textcolor{highlight2}{CryoEM $\cdot$ Biochemistry $\cdot$ Computation}
		 
		 \vspace*{-1ex}
		 \textcolor{highlight2}{Scientific communication $\cdot$ Data visualization and design}
   \end{footnotesize}
\end{scshape}
\vspace*{0.25em}

\begin{footnotesize}
   \begin{tiny}\faHome\end{tiny}~\href{https://resume.posertinlab.com}{
	   posertinlab.com
   }
   \quad \begin{tiny}\faEnvelope[regular]\end{tiny}~\href{mailto:rich@posertinlab.com}{%
	   rich@posertinlab.com
   } 
   \quad \begin{tiny}\faMobile*\end{tiny}~\href{tel:9492743495}{
	   (949) 274-3495
   }
   \quad \begin{tiny}\faGraduationCap\end{tiny}~\href{https://scholar.google.com/citations?user=eNNSP4gAAAAJ&hl=en}{
		Google Scholar
	   }

\end{footnotesize}

%%%%%%%%%%%%%%%%%%%%%%%%%%%%%%%%
%%%%%%%%%% EDUCATION %%%%%%%%%%%
%%%%%%%%%%%%%%%%%%%%%%%%%%%%%%%%
\section{Education}
\datedsubsection{2018 -- 2023}
	{%
		Vollum Institute, OHSU, Portland, OR}
	{%
		\textbf{Ph.D.}~in biochemistry and molecular biology}
	{%
	\begin{itemize}
		\item Extensive experience with cryoEM single-particle analysis, focused on small, flexible membrane proteins. Improved best map of ENaC extracellular domain from 3.1 to 2.4 \AA{}.
		\item Developed multiple functional assays and automated processing and presentation scripts for each.
		\item Developed several software tools, including \href{https://doi.org/10.1371/journal.pone.0280255}{Appia}, a chromatography suite; \href{https://github.com/PlethoraChutney/exawatcher}{Exawatcher}, a RELION job monitor; and \href{https://github.com/PlethoraChutney/Mortimer}{Mortimer}, a real-time cryoEM screening processor and organizer.
	\end{itemize}
	}

\datedsubsection{2011 -- 2015}
		{%
			Reed College, Portland, OR}
		{%
			\textbf{BA}~in biochemistry and molecular biology}
		{%
		Thesis in Janis Shampay's lab: \textit{Structural Characteristics of the PinX1 Telomerase Inhibitory Domain}}


%%%%%%%%%%%%%%%%%%%%%%%%%%%%%%%%%%
%%%%%%%%%% PUBLICATIONS %%%%%%%%%% 
%%%%%%%%%%%%%%%%%%%%%%%%%%%%%%%%%%
\section{Publications}
\newcounter{publicationCounter}			% We set a publication counter, we can continue counting the publications when switching from conferences to workshop etc.

\begin{enumerate}
	\setlength{\itemsep}{0pt}
	\item \textbf{Richard Posert}, Isabelle Baconguis. \textit{Appia: Simpler chromatography analysis and visualization}. In \textit{PLoS ONE}, 2023. doi: \href{https://doi.org/10.1371/journal.pone.0280255}{10.1371/journal.pone.0280255}
	\item Sigrid Noreng, \textbf{Richard Posert}, Arpita Bharadwaj, Alexandra Houser, Isabelle Baconguis. \textit{Molecular principles of assembly, activation, and inhibition in epithelial sodium channel}. In \textit{eLife}, 2020. doi: \href{https://doi.org/10.7554/eLife.59038}{10.7554/eLife.59038}
	\item Johannes Elferich, \textbf{Rich Posert}, Craig Yoshioka, and Eric Gouaux. \textit{HOTSPUR: A Real-time Interactive Preprocessing System for Cryo-EM Data}. In \textit{Microscopy and Microanalysis}, 2019. doi: \href{https://doi.org/10.1017/S1431927619006792}{10.1017/S1431927619006792}
	\item Sigrid Noreng, Arpita Bharadwaj, \textbf{Richard Posert}, Craig Yoshioka, Isabelle Baconguis. \textit{Structure of the human epithelial sodium channel by cryo-electron microscopy}. In \textit{eLife}. doi: \href{https://doi.org/10.7554/eLife.39340}{10.7554/eLife.39340}
	%
	\setcounter{publicationCounter}{\value{enumi}}	% Remember the counts
\end{enumerate}

% \subsection{In preparation (draft available upon request)}
% \begin{enumerate}
% 	\setcounter{enumi}{\value{publicationCounter}}
% 	\item \textbf{Richard Posert}, Arpita Bharadwaj, Isabelle Baconguis. \textit{Activating conditions of ENaC do not produce expected conformational changes}.
% \end{enumerate}

%%%%%%%%%%%%%%%%%%%%%%%%%%%%%%
%%%%%%%%%% SERVICE %%%%%%%%%%%
%%%%%%%%%%%%%%%%%%%%%%%%%%%%%%
\section{Teaching and Service}

\datedsubsection{2022 -- 2023}
{}
{CryoEM Single Particle Analysis Workshops}
{%
	I coordinated and facilitated meetings of cryoEM trainees from multiple labs at OHSU.
	During these meetings, trainees presented processing problems and we worked together to provide as many potential solutions as we could.
}

\vspace{-1em}
\datedsubsection{2022}
{Co-instructor}
{NEUS 619 - Effective Scientific Presentation}
{%
	I was one of two instructors for the course on scientific speaking and posters, which is required for first-year Neuroscience students at OHSU.
	I led classes covering topics including talk writing, slide design, audience management, and Q\&A sessions.}

\vspace{-1em}
\datedsubsection{2019 -- 2022}
{Guest lecturer}
{Biochemistry lectures}
{%
	I have been invited to give guest lectures for \textit{Chem330 - Structural Biochemistry} at Lewis and Clark and the structural biochemistry course at Reed.}

\vspace{-1em}
\datedsubsection{2018--2020}
{Lead organizer, bargaining team, executive board}
{AFSCME Local 402}
{%
	In my role as a lead organizer, I helped grow our union through one-on-one conversations from eight people to over 300, and built systems to manage that growth. I also organized, processed, and communicated the results of issue and priority surveys. As a member of the bargaining team, I was a member of a team of four that wrote and negotated a contract worth, in total, over \$3,000,000 and coordinated several teams focused on individual issues. As a member of the first e-board, I helped build a governing structure focused on equity and representation.}

\vspace{-1em}
\datedsubsection{2019}
{Teaching assistant}
{CONJ 661 - Structure and Function of Biological Molecules}
{%
	I coordinated and ran study sessions, journal clubs, and review for the required first-year structural biochemistry course at OHSU.}

%%%%%%%%%%%%%%%%%%%%%%%%%%%%%%%%%%
%%%% HONORS and SCHOLARSHIPS %%%%%
%%%%%%%%%%%%%%%%%%%%%%%%%%%%%%%%%%
\section{Awards and Funding}
\datedsubsectionnarrow{2019}
{%
	NIH T32GM071338-13}
{%
	Program in Enhanced Research Training}
{}

\datedsubsectionnarrow{2019}
{%
	AFSCME}
{%
	Emerging Leader}
{}

\end{document}
